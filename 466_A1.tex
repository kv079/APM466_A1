% Options for packages loaded elsewhere
\PassOptionsToPackage{unicode}{hyperref}
\PassOptionsToPackage{hyphens}{url}
%
\documentclass[
]{article}
\usepackage{amsmath,amssymb}
\usepackage{lmodern}
\usepackage{iftex}
\ifPDFTeX
  \usepackage[T1]{fontenc}
  \usepackage[utf8]{inputenc}
  \usepackage{textcomp} % provide euro and other symbols
\else % if luatex or xetex
  \usepackage{unicode-math}
  \defaultfontfeatures{Scale=MatchLowercase}
  \defaultfontfeatures[\rmfamily]{Ligatures=TeX,Scale=1}
\fi
% Use upquote if available, for straight quotes in verbatim environments
\IfFileExists{upquote.sty}{\usepackage{upquote}}{}
\IfFileExists{microtype.sty}{% use microtype if available
  \usepackage[]{microtype}
  \UseMicrotypeSet[protrusion]{basicmath} % disable protrusion for tt fonts
}{}
\makeatletter
\@ifundefined{KOMAClassName}{% if non-KOMA class
  \IfFileExists{parskip.sty}{%
    \usepackage{parskip}
  }{% else
    \setlength{\parindent}{0pt}
    \setlength{\parskip}{6pt plus 2pt minus 1pt}}
}{% if KOMA class
  \KOMAoptions{parskip=half}}
\makeatother
\usepackage{xcolor}
\usepackage{color}
\usepackage{fancyvrb}
\newcommand{\VerbBar}{|}
\newcommand{\VERB}{\Verb[commandchars=\\\{\}]}
\DefineVerbatimEnvironment{Highlighting}{Verbatim}{commandchars=\\\{\}}
% Add ',fontsize=\small' for more characters per line
\newenvironment{Shaded}{}{}
\newcommand{\AlertTok}[1]{\textcolor[rgb]{1.00,0.00,0.00}{\textbf{#1}}}
\newcommand{\AnnotationTok}[1]{\textcolor[rgb]{0.38,0.63,0.69}{\textbf{\textit{#1}}}}
\newcommand{\AttributeTok}[1]{\textcolor[rgb]{0.49,0.56,0.16}{#1}}
\newcommand{\BaseNTok}[1]{\textcolor[rgb]{0.25,0.63,0.44}{#1}}
\newcommand{\BuiltInTok}[1]{#1}
\newcommand{\CharTok}[1]{\textcolor[rgb]{0.25,0.44,0.63}{#1}}
\newcommand{\CommentTok}[1]{\textcolor[rgb]{0.38,0.63,0.69}{\textit{#1}}}
\newcommand{\CommentVarTok}[1]{\textcolor[rgb]{0.38,0.63,0.69}{\textbf{\textit{#1}}}}
\newcommand{\ConstantTok}[1]{\textcolor[rgb]{0.53,0.00,0.00}{#1}}
\newcommand{\ControlFlowTok}[1]{\textcolor[rgb]{0.00,0.44,0.13}{\textbf{#1}}}
\newcommand{\DataTypeTok}[1]{\textcolor[rgb]{0.56,0.13,0.00}{#1}}
\newcommand{\DecValTok}[1]{\textcolor[rgb]{0.25,0.63,0.44}{#1}}
\newcommand{\DocumentationTok}[1]{\textcolor[rgb]{0.73,0.13,0.13}{\textit{#1}}}
\newcommand{\ErrorTok}[1]{\textcolor[rgb]{1.00,0.00,0.00}{\textbf{#1}}}
\newcommand{\ExtensionTok}[1]{#1}
\newcommand{\FloatTok}[1]{\textcolor[rgb]{0.25,0.63,0.44}{#1}}
\newcommand{\FunctionTok}[1]{\textcolor[rgb]{0.02,0.16,0.49}{#1}}
\newcommand{\ImportTok}[1]{#1}
\newcommand{\InformationTok}[1]{\textcolor[rgb]{0.38,0.63,0.69}{\textbf{\textit{#1}}}}
\newcommand{\KeywordTok}[1]{\textcolor[rgb]{0.00,0.44,0.13}{\textbf{#1}}}
\newcommand{\NormalTok}[1]{#1}
\newcommand{\OperatorTok}[1]{\textcolor[rgb]{0.40,0.40,0.40}{#1}}
\newcommand{\OtherTok}[1]{\textcolor[rgb]{0.00,0.44,0.13}{#1}}
\newcommand{\PreprocessorTok}[1]{\textcolor[rgb]{0.74,0.48,0.00}{#1}}
\newcommand{\RegionMarkerTok}[1]{#1}
\newcommand{\SpecialCharTok}[1]{\textcolor[rgb]{0.25,0.44,0.63}{#1}}
\newcommand{\SpecialStringTok}[1]{\textcolor[rgb]{0.73,0.40,0.53}{#1}}
\newcommand{\StringTok}[1]{\textcolor[rgb]{0.25,0.44,0.63}{#1}}
\newcommand{\VariableTok}[1]{\textcolor[rgb]{0.10,0.09,0.49}{#1}}
\newcommand{\VerbatimStringTok}[1]{\textcolor[rgb]{0.25,0.44,0.63}{#1}}
\newcommand{\WarningTok}[1]{\textcolor[rgb]{0.38,0.63,0.69}{\textbf{\textit{#1}}}}
\setlength{\emergencystretch}{3em} % prevent overfull lines
\providecommand{\tightlist}{%
  \setlength{\itemsep}{0pt}\setlength{\parskip}{0pt}}
\setcounter{secnumdepth}{-\maxdimen} % remove section numbering
\ifLuaTeX
  \usepackage{selnolig}  % disable illegal ligatures
\fi
\IfFileExists{bookmark.sty}{\usepackage{bookmark}}{\usepackage{hyperref}}
\IfFileExists{xurl.sty}{\usepackage{xurl}}{} % add URL line breaks if available
\urlstyle{same} % disable monospaced font for URLs
\hypersetup{
  hidelinks,
  pdfcreator={LaTeX via pandoc}}

\author{}
\date{}

\begin{document}

\textbackslash title\{MAT1856/APM466 Assignment 1\}\\
\textbackslash author\{Shixun (Kevin) Huang, Student \#: 1007846869\}\\
\textbackslash date\{February, 2026\}

\textbackslash maketitle

\hypertarget{header-n200}{%
\subsubsection{\texorpdfstring{\textbf{Fundamental Questions} (25
points)}{Fundamental Questions (25 points)}}\label{header-n200}}

1.

\textbf{(a)} Governments issue bonds to raise money without
destabilizing the currency system, since printing money directly can
easily lead to inflation and weaken confidence in the currency

\textbf{(b)} The long-term yield curve might flatten if investors expect
slower economic growth and lower inflation in the future, reducing the
risk premium on long-term bonds.

\textbf{(c)} Quantitative easing is a policy where a central bank buys
large amounts of long-term assets to lower interest rates, since
COVID-19 the Fed used QE to support liquidity and stabilize financial
markets.

\begin{enumerate}
\def\labelenumi{\arabic{enumi}.}
\item
\end{enumerate}

\begin{Shaded}
\begin{Highlighting}[]
\NormalTok{CAN 4.00 Aug 26    CAN 3.00 Feb 27    CAN 2.50 Aug 27    CAN 2.25 Feb 28    CAN 3.25 Sep 28}
\NormalTok{CAN 4.00 Mar 29    CAN 3.50 Sep 29    CAN 2.75 Mar 30    CAN 2.75 Sep 30    CAN 2.75 Mar 31}
\end{Highlighting}
\end{Shaded}

I chose these 10 Government of Canada bonds to build a consistent 0-\/-5
year yield (YTM) and spot curve via bootstrapping.Because GoC bonds pay
coupons semi-annually, the ideal set would have maturities spaced about
every 6 months so each bond adds one new cash-flow date, which reduces
the need for interpolation errors. My selection keeps the issuer and
currency consistent (all GoC, CAD, Frankfurt listing) and forms an
approximately 6-month maturity ladder from 2026 to 2031 (Feb/Mar and
Aug/Sep pattern). Some maturities are not exactly 6 months apart due to
availability, but this is the closest feasible set and still reduces the
need for interpolation while covering the full 0-\/-5 year horizon.

3.

The eigenvectors of the covariance matrix represent the
\textbf{principal components}, i.e., the main orthogonal directions
along which the stochastic processes move together. Each eigenvector
describes a systematic pattern of joint variation across the points on
the curve. The corresponding eigenvalues measure how much variance is
explained by each principal component. Since the largest eigenvalues
typically account for most of the total variance, the stochastic curve
can be well-approximated by a small number of principal components.

\hypertarget{header-n214}{%
\subsubsection{\texorpdfstring{\textbf{Empirical Questions} (75
points)}{Empirical Questions (75 points)}}\label{header-n214}}

\begin{enumerate}
\def\labelenumi{\arabic{enumi}.}
\item
\end{enumerate}

(a) Daily yields to maturity (YTM) for the chosen bonds (Jan 5--19,
2026) were interpolated linearly to create 0--5 yr curves.\\
\textbf{Method:} For each day, plot YTM vs maturity and connect points
with straight lines.

{[}\\
\textbackslash text\{Interpolation: \} y(t) \textbackslash approx
y\emph{i +
\textbackslash frac\{y}\{i+1\}-y\emph{i\}\{T}\{i+1\}-T\emph{i\}(t -
T}i)\\
{]}\\
where (t) is between maturities (T\emph{i) and (T}\{i+1\}).

\textbackslash begin\{center\}\\
\textbackslash includegraphics{[}width=0.3\textbackslash textwidth{]}\{1.png\}\\
\textbackslash hspace\{1cm\} \% optional space between images\\
\textbackslash includegraphics{[}width=0.6\textbackslash textwidth{]}\{4.png\}\\
\textbackslash end\{center\}

Negative values are due to short‑maturity bonds priced high enough
(and/or with low coupons) that their yield to maturity is negative.

(b)

pseudo-code:

\begin{enumerate}
\def\labelenumi{\arabic{enumi}.}
\item
  Sort bonds by maturity.
\item
  For 1st maturity (T\emph{1): solve directly for (r(T}1)) using dirty
  price and PV formula.
\item
  For later maturities:

  \begin{itemize}
  \item
    Discount interim coupons using previously bootstrapped spot rates.
  \item
    Subtract from dirty price to get final cash‑flow PV.
  \item
    Solve for (r(T\_k)) from:

    {[}\\
    r(T\emph{k) =
    -\textbackslash frac\{\textbackslash ln(\textbackslash text\{Residual\}/CF}\textbackslash text\{final\})\}\{T\_k\}\\
    {]}
  \end{itemize}
\item
  Interpolate to get 1--5 yr spot rates.
\end{enumerate}

\textbackslash begin\{center\}\\
\textbackslash includegraphics{[}width=0.3\textbackslash textwidth{]}\{2.png\}\\
\textbackslash end\{center\}

(c)

pseudo-code:

\begin{enumerate}
\def\labelenumi{\arabic{enumi}.}
\item
  Sort bonds by maturity
\item
  For each bond i:

  \begin{itemize}
  \item
    If first bond:\\
    Solve r₁ from Price₁ = (Coupon₁ + Face) * exp(-r₁ * T₁)
  \item
    Else:\\
    Compute PV of earlier coupons using known spot rates\\
    Residual = Price\emph{i - PV}coupons\\
    Solve \$r\emph{i = -ln(Residual / (Coupon}i + Face)) / T\_i\$
  \end{itemize}
\item
  Return \$r\_i\$
\end{enumerate}

\textbackslash begin\{center\}\\
\textbackslash includegraphics{[}width=0.3\textbackslash textwidth{]}\{3.png\}\\
\textbackslash end\{center\}

\begin{enumerate}
\def\labelenumi{\arabic{enumi}.}
\item
\end{enumerate}

Covariance matrix - Yields:\\
Y1yr Y2yr Y3yr Y4yr Y5yr\\
Y1yr 0.000121 -0.000049 0.000009 -0.000011 -0.000024\\
Y2yr -0.000049 0.000123 0.000061 0.000016 0.000034\\
Y3yr 0.000009 0.000061 0.000091 0.000035 -0.000013\\
Y4yr -0.000011 0.000016 0.000035 0.000044 -0.000017\\
Y5yr -0.000024 0.000034 -0.000013 -0.000017 0.000038

Covariance matrix - Forwards:\\
F1yr-2yr F1yr-3yr F1yr-4yr F1yr-5yr\\
F1yr-2yr 0.000046 0.000038 0.000007 0.000014\\
F1yr-3yr 0.000038 0.000069 0.000027 -0.000010\\
F1yr-4yr 0.000007 0.000027 0.000033 -0.000015\\
F1yr-5yr 0.000014 -0.000010 -0.000015 0.000028

\begin{enumerate}
\def\labelenumi{\arabic{enumi}.}
\item
\end{enumerate}

Largest eigenvalue (Yield): 0.00020116951480629363

{[}-0.18732019 -0.40021488 -0.76195151 -0.46900047 0.06486642{]}

Largest eigenvalue (Forward): 0.00020116951480629363

Associated eigenvector (Forward):

{[}-0.50686408 -0.78290385 -0.35199101 0.07907396{]}

In both yield and forward rate changes over Jan 5--19, 2026, the
dominant eigenvalue corresponds to a parallel shift of the curve,
meaning that most of the changes are due to all maturity periods moving
up or down together.

\hypertarget{header-n273}{%
\subsubsection{\texorpdfstring{\textbf{References}}{References}}\label{header-n273}}

\begin{itemize}
\item
  https://www.bankofcanada.ca
\item
  https://colab.research.google.com/drive/1kCYtYmExgO7-iXjSc2Pj87BsRBJGZnp?usp=sharing
\item
  https://github.com/kv079/APM466\_A1
\item
  \textbf{Hull, J. C. (2022)} -- \emph{Options, Futures, and Other
  Derivatives}, 11th ed., Pearson.
\item
  \textbf{Fabozzi, F. J. (2016)} -- \emph{Bond Markets, Analysis and
  Strategies}, 9th ed., Pearson.
\item
  \textbf{Tuckman, B., \& Serrat, A. (2011)} -- \emph{Fixed Income
  Securities: Tools for Today's Markets}, 3rd ed., Wiley.
\item
  \textbf{Campbell, J. Y., Lo, A. W., \& MacKinlay, A. C. (1997)} --
  \emph{The Econometrics of Financial Markets}, Princeton University
  Press.
\end{itemize}

\end{document}
